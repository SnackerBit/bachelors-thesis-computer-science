\begin{figure}[!ht]
    \def\doubleFigureWidth{6.8cm}
    \def\doubleFigureHeight{4.635cm}
    \definecolor{color1a}{HTML}{bdd7e7}
    \definecolor{color2a}{HTML}{6baed6}
    \definecolor{color3a}{HTML}{2171b5}
    \definecolor{color1b}{HTML}{fcae91}
    \definecolor{color2b}{HTML}{fb6a4a}
    \definecolor{color3b}{HTML}{cb181d}
    \centering
    \captionsetup[subfigure]{oneside,margin={1.65cm,0cm}}
    \begin{subfigure}[b]{0.471\textwidth}
        \begin{tikzpicture}[scale=1]
            \begin{axis}[xlabel=$\tau$, ylabel=$\text{Re}{(\alpha(\tau))}$, title = Matsubara approximation,
                grid=both, xmin=0, xmax=1.5, ymin=-2, ymax=4, no markers, 
                every axis plot/.append style={very thick}, 
                scale only axis, height=\doubleFigureHeight, width=\doubleFigureWidth]

                \addplot[color = color1a]
                table[x=tau, y=alphas_matsubara_50_Re, col sep=space]{figures/plots/HOMPS/data/low_T_BCF.txt};
                \addlegendentry{$K = 50$}

                \addplot[color = color2a]
                table[x=tau, y=alphas_matsubara_100_Re, col sep=space]{figures/plots/HOMPS/data/low_T_BCF.txt};
                \addlegendentry{$K = 100$}

                \addplot[color = color3a]
                table[x=tau, y=alphas_matsubara_1000_Re, col sep=space]{figures/plots/HOMPS/data/low_T_BCF.txt};
                \addlegendentry{$K = 1000$}

                \addplot[color = black, dashed]
                table[x=tau, y=alpha_compare_Re, col sep=space]{figures/plots/HOMPS/data/low_T_BCF.txt};
                \addlegendentry{numerically exact}
            \end{axis}
        \end{tikzpicture}   
        \caption{}
        \label{fig:low_T_BCF_convergence_matsubara}
    \end{subfigure}\hspace{0.03\textwidth}
    \centering
    \captionsetup[subfigure]{oneside,margin={0.25cm,0cm}}
    \begin{subfigure}[b]{0.471\textwidth}
        \centering
        \begin{tikzpicture}
            \begin{axis}[xlabel=$\tau$, title = Padé approximation,
                grid=both, xmin=0, xmax=1.5, ymin=-2, ymax=4, no markers, yticklabels={,,},
                every axis plot/.append style={very thick}, 
                scale only axis, height=\doubleFigureHeight, width=\doubleFigureWidth]

                \addplot[color = color1b]
                table[x=tau, y=alphas_pade_5_Re, col sep=space]{figures/plots/HOMPS/data/low_T_BCF.txt};
                \addlegendentry{$K = 5$}

                \addplot[color = color2b]
                table[x=tau, y=alphas_pade_13_Re, col sep=space]{figures/plots/HOMPS/data/low_T_BCF.txt};
                \addlegendentry{$K = 13$}

                \addplot[color = color3b]
                table[x=tau, y=alphas_pade_30_Re, col sep=space]{figures/plots/HOMPS/data/low_T_BCF.txt};
                \addlegendentry{$K = 30$}

                \addplot[color = black, dashed]
                table[x=tau, y=alpha_compare_Re, col sep=space]{figures/plots/HOMPS/data/low_T_BCF.txt};
                \addlegendentry{numerically exact}

            \end{axis}
        \end{tikzpicture}   
        \caption{}
        \label{fig:low_T_BCF_convergence_pade}
    \end{subfigure}
    \caption{The approximation of the bath correlation function using the Debye spectral density (\ref{eq:debye_spectral_density}) 
    with $T = 0.02$, $\gamma = 5$ and $\eta = 0.5$ (low temperature, strong damping) is shown. The real part of
    the bath correlation function is approximated using the Matsubara approximation (\ref{eq:expansion_coefficients_debye_BCF_SBM_Matsubara}) and the
    Padé approximation (\ref{eq:expansion_coefficients_debye_BCF_SBM_Pade}) on the left and right respectively. The numerically exact
    result is computed by replacing the integral with a sum. One can see that the Padé approximation converges a lot faster than the Matsubara approximation.
    The imaginary part of the bath correlation function is not shown, as it is already well converged using $K=1$ terms of either approximation.}
    \label{fig:low_T_BCF_convergence} 
\end{figure}