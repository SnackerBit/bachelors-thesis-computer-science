\begin{figure}[!ht]
    \centering
    \begin{tikzpicture}[scale=1, trim axis left, trim axis right]
        \definecolor{color}{HTML}{3182bd}
        \def\singleFigureWidth{0.5\textwidth}
        \def\singleFigureHeight{0.309\textwidth}

        \begin{semilogxaxis}[xlabel=$\norm{\Psi_i}$, ylabel=$N$, area style, %xtick style={draw=none},
            ytick pos=left, ymin=0, scale only axis, height=\singleFigureHeight, width=\singleFigureWidth]

            \addplot+[ybar interval, mark=no, fill=color, color=color]
            table[x=magnitude, y=count, col sep=space]{figures/plots/HOPS/data/simple_hops_linear_magnitudes.txt};

        \end{semilogxaxis}
    \end{tikzpicture}   
    \caption{The magnitudes of states from 10000 linear HOPS realizations of the spin-boson model are shown in a histogram. There are
    big differences in the magnitudes of the different realizations, which leads to the problem that states with small magnitudes
    do not contribute much to the overall expectation value and can therefore be seen as wasted computation time.}
    \label{fig:HOPS_linear_magnitudes} 
\end{figure}