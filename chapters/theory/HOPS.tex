The NMQSD equation (\ref{eq:non_markovian_schroedinger_equation})
cannot easily be solved numerically due to the functional derivative.
However, one can bring the equation into a hierarchically structured
set of differential equations, the Hierarchy of Pure States (HOPS) \cite{Suess:2014},
which can then be integrated numerically. In this section, we will derive the linear HOPS equation, mainly
following the derivation in \cite{Hartmann:2021}, but the same result is also obtained in \cite{Suess:2014,Hartmann:2017}.
\\
We will start by approximating the bath correlation function (BCF) by a finite sum of exponentials
\begin{equation*}
    \alpha\left(\tau\right) \approx \sum_{\mu=1}^{N_\text{BCF}} \alpha_\mu\left(\tau\right) \coloneqq \sum_{\mu=1}^{N_\text{bath}} g_\mu e^{-\omega_\mu\tau},
\end{equation*}
with constants $g_\mu$ and $\omega_\mu$. The total number of terms $N_\text{BCF}$ corresponds
to the number of harmonic oscillators coupling to each site of the system. Such an approximation
of the bath correlation function is possible for many systems of interest, and a specific
example is given in section \ref{}, where we approximate the BCF of the spin-boson model.
\\
In the following, we will assume that the system lives only on a single site, $N=1$. The derivation of the HOPS can
trivially be expanded to $N>1$ sites.
\\
We will work with the discrete version of the NMQSD equation (\ref{eq:non_markovian_schroedinger_equation}) \cite{Hartmann:2021}
\begin{equation}
    \label{eq:discrete_NMQSD_equation}
    \Psi_{n+1} = \Psi_n + \Delta \cdot \left\{
        -iH_\text{S}+\left[
            Lz_{n}^* - L^\dagger \sum_{m=0}^{n-1} \alpha\left(n\Delta-m\Delta\right)\frac{\partial}{\partial z_{m}^*}
        \right]
    \right\} \Psi_n,
\end{equation}
where we introduced the time step $\Delta$, with $t \equiv n\cdot\Delta$, and $z^* = \left\{z_{1}^*, z_{2}^*,\dots,z_{n}^*\right\}$
now is a discrete stochastic process. One can easily see that the NMQSD equation
is recovered if the limit $\Delta\rightarrow0$ is taken. The reason for using the discrete version of the NMQSD equation
is that we can replace the functional derivative with an ordinary derivative
and the derivation becomes much more intuitive.
\\
We define the operator
\begin{equation*}
    D_\mu^n \coloneqq \sum_{m=0}^{n-1} \alpha\left(n\Delta - m\Delta\right) \frac{\partial}{\partial z_m^*}
    = g_\mu \sum_{m=0}^{n-1} e^{-\omega_\mu\left(n-m\right)\cdot\Delta}
    \frac{\partial}{\partial z_m^*}
\end{equation*}
and the auxillary states
\begin{equation}
    \label{eq:definition_of_auxillary_states}
    \Psi_n^{(\vec{k})} \coloneqq \prod_{\mu=1}^{N_\text{BCF}} \left(D_\mu^n\right)^{k_\mu}\Psi_n,
\end{equation}
using an index vector $\vec{k}\in\mathbb{N}_0^{N_\text{BCF}}$. The physical pure state is
recovered when setting the index vector to zero, $\Psi_n = \Psi_n^{(\vec{0})}$.
Using these definitions, we can rewrite the discrete NMQSD equation (\ref{eq:discrete_NMQSD_equation}):
\begin{equation*}
    \Psi_{n+1}^{(\vec{0})} = \Psi_{n}^{(\vec{0})} + \Delta \cdot \left(
        -iH_\text{S} + Lz_n^* - L^\dagger\sum_{\mu=1}^{N_\text{BCF}}D_\mu^n
    \right) \Psi_{n}^{(\vec{0})}
    = \Psi_{n}^{(\vec{0})} + \Delta \cdot \left(
        -iH_\text{S} + Lz_n^*
    \right) \Psi_{n}^{(\vec{0})} - L^\dagger\sum_{\mu=1}^{N_\text{BCF}}\Psi_n^{(\vec{0}+\hat{e}_\mu)},
\end{equation*}
where $\hat{e}_\mu$ is the $\mu$-th unit vector.
\\
Our next goal is to derive an equation of motion for an arbitrary auxillary state $\Psi_n^{(\vec{k})}$.
Using equation (\ref{eq:definition_of_auxillary_states}) we can write
\begin{equation}
    \label{eq:auxillary_state_np1}
    \Psi_{n+1}^{(\vec{k})} = \prod_{\mu=1}^{N_\text{BCF}} \left(D_\mu^{n+1}\right)^{k_\mu} \Psi_{n+1}.
\end{equation}
We can expand
\begin{equation*}
    D_\mu^{n+1} = \left(1-\omega_\mu\cdot\Delta\right)\left(g_\mu\frac{\partial}{\partial z_n^*} + D_\mu^n\right) + O\left(\Delta^2\right)
\end{equation*}
and hence
\begin{equation*}
    \left(D_\mu^{n+1}\right)^{k_\mu} = 
    \left(1-k_\mu\omega_\mu\Delta\right)
    \left(g_\mu\frac{\partial}{\partial z_n^*} + D_\mu^n\right)^{k_\mu} + O\left(\Delta^2\right).
\end{equation*}
To further simplify eq (\ref{eq:auxillary_state_np1}), we can use the fact that the physical pure state
at time $t$, $\Psi_n$, depends only on the stochastic variables $z_1^*, z_2^*, \dots, z_{n-1}^*$, but not
on $z_n^*$, which we can write as $\Psi_n=\Psi_n\left(z^*|_0^{n-1}\right)$. It follows that
$\frac{\partial}{\partial z_n^*}\Psi_n = 0$. Using equation (\ref{eq:discrete_NMQSD_equation})
we can see $\frac{\partial^2}{\partial z_n^{*2}}\Psi_{n+1} = 0$ and therefore
\begin{equation}
    \label{eq:simplification_of_D_psi_np1}
    \left(D_\mu^{n+1}\right)^{k_\mu} \Psi_{n+1} = \left(1-k_\mu\omega_\mu\Delta\right)
    \left(k_\mu g_\mu \left(D_\mu^n\right)^{k_\mu-1}\frac{\partial}{\partial z_n^*} + \left(D_\mu^n\right)^{k_\mu}\right) + O\left(\Delta^2\right).
\end{equation}
Inserting equations (\ref{eq:discrete_NMQSD_equation}) and (\ref{eq:simplification_of_D_psi_np1}) into equation (\ref{eq:auxillary_state_np1}) and performing some
additional algebra, one arrives at
\begin{equation*}
    \Psi_{n+1}^{(\vec{k})} = \Psi_n^{(\vec{k})} + \Delta\cdot\left(
        -iH_\text{S} - \vec{k}\cdot\vec{\omega} + z_t^*L
    \right) \Psi_n^(\vec{k}) 
    + \Delta \cdot L\sum_{\mu=1}^{N_\text{BCF}}g_\mu\Psi_n^{(\vec{k}-\hat{e}_\mu)}
    - \Delta \cdot L^\dagger\sum_{\mu=1}^{N_\text{BCF}}\Psi_n^{(\vec{k}+\hat{e}_\mu)}.
\end{equation*}
Performing the limit $\Delta \rightarrow 0$, we obtain the linear HOPS equation for a single 
site coupled to multiple heat baths:
\begin{equation}
    \label{eq:linear_HOPS_single_site}
    \frac{\partial}{\partial t}\Psi_t^{(\vec{k})} = \left(
        -iH_\text{S} - \vec{k}\cdot\vec{\omega} + z_t^*L
    \right) \Psi_t^{(\vec{k})}
    + L\sum_{\mu=1}^{N_\text{BCF}}g_\mu\Psi_t^{(\vec{k}-\hat{e}_\mu)}
    - L^\dagger\sum_{\mu=1}^{N_\text{BCF}}\Psi_t^{(\vec{k}+\hat{e}_\mu)}.
\end{equation}
Repeating the derivation for $N>1$ sites yields
\begin{equation}
    \label{eq:linear_HOPS_multiple_sites}
    \frac{\partial}{\partial t}\Psi_t^{(\vec{k})} = \left(
        -iH_\text{S} - \vec{k}\cdot\vec{\omega} + \sum_{i=1}^{N} z_{i,t}^*L_i
    \right) \Psi_t^{(\vec{k})}
    + \sum_{i=1}^{N} \left[
        L_i\sum_{\mu=1}^{N_\text{BCF}}g_\mu\Psi_t^{(\vec{k}-\hat{e}_\mu)}
        - L_i^\dagger\sum_{\mu=1}^{N_\text{BCF}}\Psi_t^{(\vec{k}+\hat{e}_\mu)}
    \right].
\end{equation}
Note that the HOPS equations do not contain functional derivatives
and therefore can be integrated numerically.