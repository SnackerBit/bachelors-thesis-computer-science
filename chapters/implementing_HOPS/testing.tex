To test my implementation of the HOPS, I use the spin boson model (\ref{eq:system_hamiltonian_SBM}) with $\Delta = 1$ and $\epsilon = 0$. 
For the bath correlation function I use the simple expansion
\begin{equation*}
    \alpha(\tau) = ge^{-\omega\tau}
\end{equation*}
with only a single bath mode and constants $g = 2$ and $\omega = 0.5 + 2i$. The constants for the spin-boson model
and the bath correlation function are taken from \cite{Suess:2014}. In figure \ref{fig:full_runs_HOPS}, the dynamics
of the model is shown, computed using 100, 1000, and 10000 realizations of the stochastic process. Both linear and non-linear 
HOPS are shown. One can directly see that the linear HOPS converges much slower. The reason for this is that the difference in 
magnitudes of the states from different realizations of the stochastic process can be really large (see figure \ref{fig:HOPS_linear_magnitudes}).
Because of this, most realizations do not contribute much to the stochastic expectation value, which is therefore dominated
by only a few realizations with large magnitudes. In the non-linear HOPS, states are normalized, and thus each realization
is weighted equally in the stochastic expectation value, leading to faster convergence.\\
  


\begin{figure}[!ht]
    \def\doubleFigureWidth{6.8cm}
    \def\doubleFigureHeight{4.635cm}
    \definecolor{color1}{HTML}{bdd7e7}
    \definecolor{color2}{HTML}{6baed6}
    \definecolor{color3}{HTML}{2171b5}
    \centering
    \begin{subfigure}[b]{0.471\textwidth}
        \begin{tikzpicture}[scale=1]
            \begin{axis}[xlabel=$t$, ylabel=$\mathbb{E}\left\lbrack\langle\hat{\sigma}_z\rangle\right\rbrack$, title=linear HOPS,
                grid=both, xmin=0, xmax=50, ymin=-1, ymax=1, no markers, 
                every axis plot/.append style={very thick}, 
                scale only axis, height=\doubleFigureHeight, width=\doubleFigureWidth]

                \addplot[color = color1]
                table[x=t, y=sigma_z_100, col sep=space]{figures/plots/HOPS/data/simple_hops_linear_multiple_realizations.txt};
                %\addlegendentry{100 realizations}

                \addplot[color = color2]
                table[x=t, y=sigma_z_1000, col sep=space]{figures/plots/HOPS/data/simple_hops_linear_multiple_realizations.txt};
                %\addlegendentry{1000 realizations}

                \addplot[color = color3]
                table[x=t, y=sigma_z_10000, col sep=space]{figures/plots/HOPS/data/simple_hops_linear_multiple_realizations.txt};
                %\addlegendentry{10000 realizations}
            \end{axis}
        \end{tikzpicture}   
    \end{subfigure}\hspace{0.05\textwidth}
    \centering
    \begin{subfigure}[b]{0.471\textwidth}
        \centering
        \begin{tikzpicture}
            \begin{axis}[xlabel=$t$, yticklabels={,,}, legend pos=south east, title=non-linear HOPS,
                grid=both, xmin=0, xmax=50, ymin=-1, ymax=1, no markers, 
                every axis plot/.append style={very thick}, 
                scale only axis, height=\doubleFigureHeight, width=\doubleFigureWidth]

                \addplot[color = color1]
                table[x=t, y=sigma_z_100, col sep=space]{figures/plots/HOPS/data/simple_hops_nonlinear_multiple_realizations.txt};
                \addlegendentry{100 realizations}

                \addplot[color = color2]
                table[x=t, y=sigma_z_1000, col sep=space]{figures/plots/HOPS/data/simple_hops_nonlinear_multiple_realizations.txt};
                \addlegendentry{1000 realizations}

                \addplot[color = color3]
                table[x=t, y=sigma_z_10000, col sep=space]{figures/plots/HOPS/data/simple_hops_nonlinear_multiple_realizations.txt};
                \addlegendentry{10000 realizations}
            \end{axis}
        \end{tikzpicture}   
    \end{subfigure}
    \caption{The dynamics of the spin-boson model are computed using the HOPS method. The stochastic expectation value of the $\hat{\sigma}_z$ operator is
    plotted against the time. The expectation value is taken over 100, 1000, and 10000 realizations of the stochastic processes. Both the linear (left) and the non-linear (right)
    HOPS was used. The time steps for all realizations was chosen as $\Delta t = 0.05$. The parameters for the spin-boson model are $\Delta = 1$ and $\epsilon = 0$. The  HOPS was truncated using simple truncation (\ref{eq:simple_truncation}) with $N_\text{trunc} = 8$.}
    \label{fig:full_runs_HOPS} 
\end{figure}
\begin{figure}[!ht]
    \centering
    \begin{tikzpicture}[scale=1, trim axis left, trim axis right]
        \definecolor{color}{HTML}{3182bd}
        \def\singleFigureWidth{0.5\textwidth}
        \def\singleFigureHeight{0.309\textwidth}

        \begin{semilogxaxis}[xlabel=$\norm{\Psi_i}$, ylabel=$N$, area style, %xtick style={draw=none},
            ytick pos=left, ymin=0, scale only axis, height=\singleFigureHeight, width=\singleFigureWidth]

            \addplot+[ybar interval, mark=no, fill=color, color=color]
            table[x=magnitude, y=count, col sep=space]{figures/plots/HOPS/data/simple_hops_linear_magnitudes.txt};

        \end{semilogxaxis}
    \end{tikzpicture}   
    \caption{The magnitudes of states from 10000 linear HOPS realizations of the spin-boson model are shown in a histogram. There are
    big differences in the magnitudes of the different realizations, which leads to the problem that states with small magnitudes
    do not contribute much to the overall expectation value and can therefore be seen as wasted computation time.}
    \label{fig:HOPS_linear_magnitudes} 
\end{figure}
\begin{figure}[!ht]
    \centering
    \begin{tikzpicture}[scale=1, trim axis left, trim axis right]
        \definecolor{color1}{HTML}{bdd7e7}
        \definecolor{color2}{HTML}{6baed6}
        \definecolor{color3}{HTML}{3182bd}
        \definecolor{color4}{HTML}{08519c}
        \def\singleFigureWidth{0.5\textwidth}
        \def\singleFigureHeight{0.309\textwidth}

        \begin{axis}[xlabel=$t$, ylabel=$\mathbb{E}\left\lbrack\langle\hat{\sigma}_z\rangle\right\rbrack$,
            grid=both, xmin=0, xmax=20, ymin=-0.2, ymax=1, no markers, 
            every axis plot/.append style={very thick}, scale only axis, height=\singleFigureHeight, width=\singleFigureWidth]

            \addplot[color = color1]
            table[x=t, y=sigma_z_N_trunc_2, col sep=space]{figures/plots/HOPS/data/simple_hops_nonlinear_N_trunc_convergence.txt};
            \addlegendentry{$N_\text{trunc} = 2$}

            \addplot[color = color2]
            table[x=t, y=sigma_z_N_trunc_4, col sep=space]{figures/plots/HOPS/data/simple_hops_nonlinear_N_trunc_convergence.txt};
            \addlegendentry{$N_\text{trunc} = 4$}

            \addplot[color = color3]
            table[x=t, y=sigma_z_N_trunc_8, col sep=space]{figures/plots/HOPS/data/simple_hops_nonlinear_N_trunc_convergence.txt};
            \addlegendentry{$N_\text{trunc} = 8$}

            \addplot[color = color4]
            table[x=t, y=sigma_z_N_trunc_16, col sep=space]{figures/plots/HOPS/data/simple_hops_nonlinear_N_trunc_convergence.txt};
            \addlegendentry{$N_\text{trunc} = 16$}

        \end{axis}
    \end{tikzpicture}   
    \caption{In this figure, the convergence of non-linear HOPS in the $N_\text{trunc}$ parameter is shown. For different values of
    $N_\text{trunc}$, 10000 realizations of non-linear HOPS with the spin-boson model were computed each. The other parameters are the same
    as for Figure \ref{fig:full_runs_HOPS}. $N_\text{trunc} = 8$, which was used for the full runs in Figure \ref{fig:full_runs_HOPS}, is already well converged.}
    \label{fig:HOPS_nonlinear_convergence} 
\end{figure}