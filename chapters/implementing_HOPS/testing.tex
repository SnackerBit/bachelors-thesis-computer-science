To test my implementation of the HOPS, I use the spin-boson model (\ref{eq:system_hamiltonian_SBM}) with $\Delta = 1$ and $\epsilon = 0$. 
For the bath correlation function I use the simple expansion
\begin{equation}
    \label{eq:simple_hops_BCF}
    \alpha(\tau) = ge^{-\omega\tau}
\end{equation}
with only a single bath mode and constants $g = 2$ and $\omega = 0.5 + 2i$. The constants for the spin-boson model
and the bath correlation function are taken from \cite{Suess:2014}. I begin by testing the generation of the stochastic
process. For this, 100, 1000, and 10000 realizations of the stochastic process are generated. The expectation value
$\mathbb{E}{[z_tz_0]}$ is then computed and shown in Figure \ref{fig:stochastic_processes}.
One can see that the expectation value approaches the bath correlation function as the number of realizations is increased,
meaning Condition (\ref{eq:stochastic_process_condition}) is well fulfilled.
\begin{figure}[!ht]
    \def\doubleFigureWidth{6.8cm}
    \def\doubleFigureHeight{4.635cm}
    \definecolor{color1}{HTML}{bdd7e7}
    \definecolor{color2}{HTML}{6baed6}
    \definecolor{color3}{HTML}{2171b5}
    \centering
    \begin{subfigure}[b]{0.431\textwidth}
        \begin{tikzpicture}[scale=1]
            \begin{axis}[xlabel=$t$, title=$\text{Re}\left(\mathbb{E}{[z_tz_0^*]}\right)$,
                no markers, every axis plot/.append style={very thick}, scale only axis, height=\doubleFigureHeight, width=\doubleFigureWidth,
                ymin=-2, ymax=2.5, xmin=0, xmax=10, grid=both]
                \addplot[color = color1]
                table[x=t, y=z_t_z_0_c_Re, col sep=space]{figures/plots/HOPS/data/stochastic_processes_100.txt};
                \addlegendentry{100 realizations}
                \addplot[color = color2]
                table[x=t, y=z_t_z_0_c_Re, col sep=space]{figures/plots/HOPS/data/stochastic_processes_1000.txt};
                \addlegendentry{1000 realizations}
                \addplot[color = color3]
                table[x=t, y=z_t_z_0_c_Re, col sep=space]{figures/plots/HOPS/data/stochastic_processes_10000.txt};
                \addlegendentry{10000 realizations}
                %\addplot[color = color5]
                %table[x=t, y=z_t_z_0_c_Re, col sep=space]{figures/plots/HOPS/data/stochastic_processes_100000.txt};
                %\addlegendentry{100000 realizations}
                \addplot[color = black, dashed]
                table[x=t, y=alpha_Re, col sep=space]{figures/plots/HOPS/data/stochastic_processes_100.txt};
                \addlegendentry{$\alpha(t)$}
            \end{axis}
        \end{tikzpicture}   
    \end{subfigure}\hspace{0.05\textwidth}
    \centering
    \begin{subfigure}[b]{0.431\textwidth}
        \centering
        \begin{tikzpicture}[scale=1]
            \begin{axis}[xlabel=$t$, title=$\text{Im}\left(\mathbb{E}{[z_tz_0^*]}\right)$, yticklabels={,,},
                no markers, every axis plot/.append style={very thick}, scale only axis, height=\doubleFigureHeight, width=\doubleFigureWidth,
                ymin=-2, ymax=2.5, xmin=0, xmax=10, grid=both]
                \addplot[color = color1]
                table[x=t, y=z_t_z_0_c_Im, col sep=space]{figures/plots/HOPS/data/stochastic_processes_100.txt};
                %\addlegendentry{100 realizations}
                \addplot[color = color2]
                table[x=t, y=z_t_z_0_c_Im, col sep=space]{figures/plots/HOPS/data/stochastic_processes_1000.txt};
                %\addlegendentry{1000 realizations}
                \addplot[color = color3]
                table[x=t, y=z_t_z_0_c_Im, col sep=space]{figures/plots/HOPS/data/stochastic_processes_10000.txt};
                %\addlegendentry{10000 realizations}
                %\addplot[color = color5]
                %table[x=t, y=z_t_z_0_c_Im, col sep=space]{figures/plots/HOPS/data/stochastic_processes_100000.txt};
                %\addlegendentry{100000 realizations}
                \addplot[color = black, dashed]
                table[x=t, y=alpha_Im, col sep=space]{figures/plots/HOPS/data/stochastic_processes_100.txt};
                %\addlegendentry{$\alpha(t)$}
            \end{axis}
        \end{tikzpicture}   
    \end{subfigure}
    \caption{The real and imaginary parts of the expectation value $\mathbb{E}{[z_tz_0^*]}$ of the stochastic process used for
    testing HOPS with the spin-boson model are computed from 100, 1000, and 10000 realizations. One can see that the correlations
    approach the bath correlation function $\alpha(\tau)$ (\ref{eq:simple_hops_BCF}) with $g = 2$ and $\omega = 0.5 + 2i$.}
    \label{fig:stochastic_processes} 
\end{figure}
\begin{figure}[!ht]
    \def\doubleFigureWidth{6.8cm}
    \def\doubleFigureHeight{4.635cm}
    \definecolor{color1}{HTML}{bdd7e7}
    \definecolor{color2}{HTML}{6baed6}
    \definecolor{color3}{HTML}{2171b5}
    \centering
    \begin{subfigure}[b]{0.471\textwidth}
        \begin{tikzpicture}[scale=1]
            \begin{axis}[xlabel=$t$, ylabel=$\mathbb{E}\left\lbrack\langle\hat{\sigma}_z\rangle\right\rbrack$, title=linear HOPS,
                grid=both, xmin=0, xmax=50, ymin=-1, ymax=1, no markers, 
                every axis plot/.append style={very thick}, 
                scale only axis, height=\doubleFigureHeight, width=\doubleFigureWidth]

                \addplot[color = color1]
                table[x=t, y=sigma_z_100, col sep=space]{figures/plots/HOPS/data/simple_hops_linear_multiple_realizations.txt};
                %\addlegendentry{100 realizations}

                \addplot[color = color2]
                table[x=t, y=sigma_z_1000, col sep=space]{figures/plots/HOPS/data/simple_hops_linear_multiple_realizations.txt};
                %\addlegendentry{1000 realizations}

                \addplot[color = color3]
                table[x=t, y=sigma_z_10000, col sep=space]{figures/plots/HOPS/data/simple_hops_linear_multiple_realizations.txt};
                %\addlegendentry{10000 realizations}
            \end{axis}
        \end{tikzpicture}   
    \end{subfigure}\hspace{0.05\textwidth}
    \centering
    \begin{subfigure}[b]{0.471\textwidth}
        \centering
        \begin{tikzpicture}
            \begin{axis}[xlabel=$t$, yticklabels={,,}, legend pos=south east, title=non-linear HOPS,
                grid=both, xmin=0, xmax=50, ymin=-1, ymax=1, no markers, 
                every axis plot/.append style={very thick}, 
                scale only axis, height=\doubleFigureHeight, width=\doubleFigureWidth]

                \addplot[color = color1]
                table[x=t, y=sigma_z_100, col sep=space]{figures/plots/HOPS/data/simple_hops_nonlinear_multiple_realizations.txt};
                \addlegendentry{100 realizations}

                \addplot[color = color2]
                table[x=t, y=sigma_z_1000, col sep=space]{figures/plots/HOPS/data/simple_hops_nonlinear_multiple_realizations.txt};
                \addlegendentry{1000 realizations}

                \addplot[color = color3]
                table[x=t, y=sigma_z_10000, col sep=space]{figures/plots/HOPS/data/simple_hops_nonlinear_multiple_realizations.txt};
                \addlegendentry{10000 realizations}
            \end{axis}
        \end{tikzpicture}   
    \end{subfigure}
    \caption{The dynamics of the spin-boson model are computed using the HOPS method. The stochastic expectation value of the $\hat{\sigma}_z$ operator is
    plotted against the time. The expectation value is taken over 100, 1000, and 10000 realizations of the stochastic processes. Both the linear (left) and the non-linear (right)
    HOPS was used. The time steps for all realizations was chosen as $\Delta t = 0.05$. The parameters for the spin-boson model are $\Delta = 1$ and $\epsilon = 0$. The  HOPS was truncated using simple truncation (\ref{eq:simple_truncation}) with $N_\text{trunc} = 8$.}
    \label{fig:full_runs_HOPS} 
\end{figure}
\newpage
\noindent Next, I integrate both the linear and non-linear HOPS as detailed in Section \ref{sec:implementing_HOPS}. I use $N_\text{trunc} = 8$
and a time step of $\Delta t = 0.02$. The expectation value
$\mathbb{E}\left\lbrack\langle\hat{\sigma}_z\rangle\right\rbrack$ is then computed by averaging over 100, 1000, and 1000 realizations.
The result is shown in Figure \ref{fig:full_runs_HOPS}.
One can directly see that the linear HOPS converges much slower than the non-linear HOPS. The reason for this is that the states from different
realizations of the noise have strongly different magnitudes, as can be seen in Figure \ref{fig:HOPS_linear_magnitudes}. When computing the stochastic expectation value for linear HOPS (\ref{eq:stochastic_expectation_value_linear}),
states with small magnitudes do not contribute much to the result. The expectation value is thus dominated by only a few states
with large magnitudes, which leads to slow convergence. In contrast, the states in the non-linear
HOPS are normalized, avoiding the problem and leading to faster convergence. \\
The convergence of the non-linear HOPS method with respect to the truncation dimension $N_\text{trunc}$ is shown in Figure \ref{fig:HOPS_nonlinear_convergence}.\\
My results match well the ones obtained in \cite{Suess:2014}.
\begin{figure}[!ht]
    \centering
    \begin{tikzpicture}[scale=1, trim axis left, trim axis right]
        \definecolor{color}{HTML}{3182bd}
        \def\singleFigureWidth{0.5\textwidth}
        \def\singleFigureHeight{0.309\textwidth}

        \begin{semilogxaxis}[xlabel=$\norm{\Psi_i}$, ylabel=$N$, area style, %xtick style={draw=none},
            ytick pos=left, ymin=0, scale only axis, height=\singleFigureHeight, width=\singleFigureWidth]

            \addplot+[ybar interval, mark=no, fill=color, color=color]
            table[x=magnitude, y=count, col sep=space]{figures/plots/HOPS/data/simple_hops_linear_magnitudes.txt};

        \end{semilogxaxis}
    \end{tikzpicture}   
    \caption{The magnitudes of states from 10000 linear HOPS realizations of the spin-boson model are shown in a histogram. There are
    big differences in the magnitudes of the different realizations, which leads to the problem that states with small magnitudes
    do not contribute much to the overall expectation value and can therefore be seen as wasted computation time.}
    \label{fig:HOPS_linear_magnitudes} 
\end{figure}
\begin{figure}[!ht]
    \centering
    \begin{tikzpicture}[scale=1, trim axis left, trim axis right]
        \definecolor{color1}{HTML}{bdd7e7}
        \definecolor{color2}{HTML}{6baed6}
        \definecolor{color3}{HTML}{3182bd}
        \definecolor{color4}{HTML}{08519c}
        \def\singleFigureWidth{0.5\textwidth}
        \def\singleFigureHeight{0.309\textwidth}

        \begin{axis}[xlabel=$t$, ylabel=$\mathbb{E}\left\lbrack\langle\hat{\sigma}_z\rangle\right\rbrack$,
            grid=both, xmin=0, xmax=20, ymin=-0.2, ymax=1, no markers, 
            every axis plot/.append style={very thick}, scale only axis, height=\singleFigureHeight, width=\singleFigureWidth]

            \addplot[color = color1]
            table[x=t, y=sigma_z_N_trunc_2, col sep=space]{figures/plots/HOPS/data/simple_hops_nonlinear_N_trunc_convergence.txt};
            \addlegendentry{$N_\text{trunc} = 2$}

            \addplot[color = color2]
            table[x=t, y=sigma_z_N_trunc_4, col sep=space]{figures/plots/HOPS/data/simple_hops_nonlinear_N_trunc_convergence.txt};
            \addlegendentry{$N_\text{trunc} = 4$}

            \addplot[color = color3]
            table[x=t, y=sigma_z_N_trunc_8, col sep=space]{figures/plots/HOPS/data/simple_hops_nonlinear_N_trunc_convergence.txt};
            \addlegendentry{$N_\text{trunc} = 8$}

            \addplot[color = color4]
            table[x=t, y=sigma_z_N_trunc_16, col sep=space]{figures/plots/HOPS/data/simple_hops_nonlinear_N_trunc_convergence.txt};
            \addlegendentry{$N_\text{trunc} = 16$}

        \end{axis}
    \end{tikzpicture}   
    \caption{In this figure, the convergence of non-linear HOPS in the $N_\text{trunc}$ parameter is shown. For different values of
    $N_\text{trunc}$, 10000 realizations of non-linear HOPS with the spin-boson model were computed each. The other parameters are the same
    as for Figure \ref{fig:full_runs_HOPS}. $N_\text{trunc} = 8$, which was used for the full runs in Figure \ref{fig:full_runs_HOPS}, is already well converged.}
    \label{fig:HOPS_nonlinear_convergence} 
\end{figure}