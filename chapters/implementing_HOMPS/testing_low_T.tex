As a second test, I use a low temperature and high damping case of the spin-boson model (\ref{eq:system_hamiltonian_SBM}). I again use the Debye spectral
density and set the parameters to $T = 0.02$, $\gamma = 5$, $\eta = 0.5$, $\epsilon = 2$, $\Delta = -2$, which are also used in \cite{Suess:2014,Song:2016}.
The approximations of the bath correlation function using the Matsubara summation and Padé approximation methods are shown in Figures \ref{fig:low_T_BCF_convergence_matsubara} and \ref{fig:low_T_BCF_convergence_pade} respectively.
At low temperatures the Padé approximation converges a lot faster than the Matsubara summation. The following computations are done with $K=13$ terms of the
Padé approximation, which yields a sufficiently converged BCF.
\begin{figure}[!ht]
    \def\doubleFigureWidth{6.8cm}
    \def\doubleFigureHeight{4.635cm}
    \definecolor{color1a}{HTML}{bdd7e7}
    \definecolor{color2a}{HTML}{6baed6}
    \definecolor{color3a}{HTML}{2171b5}
    \definecolor{color1b}{HTML}{fcae91}
    \definecolor{color2b}{HTML}{fb6a4a}
    \definecolor{color3b}{HTML}{cb181d}
    \centering
    \captionsetup[subfigure]{oneside,margin={1.65cm,0cm}}
    \begin{subfigure}[b]{0.471\textwidth}
        \begin{tikzpicture}[scale=1]
            \begin{axis}[xlabel=$\tau$, ylabel=$\text{Re}{(\alpha(\tau))}$, title = Matsubara approximation,
                grid=both, xmin=0, xmax=1.5, ymin=-2, ymax=4, no markers, 
                every axis plot/.append style={very thick}, 
                scale only axis, height=\doubleFigureHeight, width=\doubleFigureWidth]

                \addplot[color = color1a]
                table[x=tau, y=alphas_matsubara_50_Re, col sep=space]{figures/plots/HOMPS/data/low_T_BCF.txt};
                \addlegendentry{$K = 50$}

                \addplot[color = color2a]
                table[x=tau, y=alphas_matsubara_100_Re, col sep=space]{figures/plots/HOMPS/data/low_T_BCF.txt};
                \addlegendentry{$K = 100$}

                \addplot[color = color3a]
                table[x=tau, y=alphas_matsubara_1000_Re, col sep=space]{figures/plots/HOMPS/data/low_T_BCF.txt};
                \addlegendentry{$K = 1000$}

                \addplot[color = black, dashed]
                table[x=tau, y=alpha_compare_Re, col sep=space]{figures/plots/HOMPS/data/low_T_BCF.txt};
                \addlegendentry{numerically exact}
            \end{axis}
        \end{tikzpicture}   
        \caption{}
        \label{fig:low_T_BCF_convergence_matsubara}
    \end{subfigure}\hspace{0.03\textwidth}
    \centering
    \captionsetup[subfigure]{oneside,margin={0.25cm,0cm}}
    \begin{subfigure}[b]{0.471\textwidth}
        \centering
        \begin{tikzpicture}
            \begin{axis}[xlabel=$\tau$, title = Padé approximation,
                grid=both, xmin=0, xmax=1.5, ymin=-2, ymax=4, no markers, yticklabels={,,},
                every axis plot/.append style={very thick}, 
                scale only axis, height=\doubleFigureHeight, width=\doubleFigureWidth]

                \addplot[color = color1b]
                table[x=tau, y=alphas_pade_5_Re, col sep=space]{figures/plots/HOMPS/data/low_T_BCF.txt};
                \addlegendentry{$K = 5$}

                \addplot[color = color2b]
                table[x=tau, y=alphas_pade_13_Re, col sep=space]{figures/plots/HOMPS/data/low_T_BCF.txt};
                \addlegendentry{$K = 13$}

                \addplot[color = color3b]
                table[x=tau, y=alphas_pade_30_Re, col sep=space]{figures/plots/HOMPS/data/low_T_BCF.txt};
                \addlegendentry{$K = 30$}

                \addplot[color = black, dashed]
                table[x=tau, y=alpha_compare_Re, col sep=space]{figures/plots/HOMPS/data/low_T_BCF.txt};
                \addlegendentry{numerically exact}

            \end{axis}
        \end{tikzpicture}   
        \caption{}
        \label{fig:low_T_BCF_convergence_pade}
    \end{subfigure}
    \caption{The approximation of the bath correlation function using the Debye spectral density (\ref{eq:debye_spectral_density}) 
    with $T = 0.02$, $\gamma = 5$ and $\eta = 0.5$ (low temperature, strong damping) is shown. The real part of
    the bath correlation function is approximated using the Matsubara approximation (\ref{eq:expansion_coefficients_debye_BCF_SBM_Matsubara}) and the
    Padé approximation (\ref{eq:expansion_coefficients_debye_BCF_SBM_Pade}) on the left and right respectively. The numerically exact
    result is computed by replacing the integral with a sum. One can see that the Padé approximation converges a lot faster than the Matsubara approximation.
    The imaginary part of the bath correlation function is not shown, as it is already well converged using $K=1$ terms of either approximation.}
    \label{fig:low_T_BCF_convergence} 
\end{figure}
\begin{figure}[H]
    \def\doubleFigureWidth{6.8cm}
    \def\doubleFigureHeight{4.635cm}
    \definecolor{color1}{HTML}{bdd7e7}
    \definecolor{color2}{HTML}{6baed6}
    \definecolor{color3}{HTML}{2171b5}
    \centering
    \begin{subfigure}[b]{0.471\textwidth}
        \begin{tikzpicture}[scale=1]
            \begin{axis}[xlabel=$t$, ylabel=$\mathbb{E}\left\lbrack\langle\hat{\sigma}_z\rangle\right\rbrack$,
                grid=both, xmin=0, xmax=30, ymin=-1, ymax=1, no markers, 
                every axis plot/.append style={very thick}, 
                scale only axis, height=\doubleFigureHeight, width=\doubleFigureWidth, ylabel near ticks, ylabel shift={-10pt}]

                \addplot[color = color1]
                table[x=t, y=sigma_zs_1em3, col sep=space]{figures/plots/HOMPS/data/homps_low_T_eps_convergence.txt};
                \addlegendentry{$\epsilon_\text{SVD} = 10^{-3}$}

                \addplot[color = color2]
                table[x=t, y=sigma_zs_1em4, col sep=space]{figures/plots/HOMPS/data/homps_low_T_eps_convergence.txt};
                \addlegendentry{$\epsilon_\text{SVD} = 10^{-4}$}

                \addplot[color = color3]
                table[x=t, y=sigma_zs_1em5, col sep=space]{figures/plots/HOMPS/data/homps_low_T_eps_convergence.txt};
                \addlegendentry{$\epsilon_\text{SVD} = 10^{-5}$}
            \end{axis}
        \end{tikzpicture}
    \end{subfigure}\hspace{0.02\textwidth}
    \centering
    \begin{subfigure}[b]{0.471\textwidth}
        \centering
        \begin{tikzpicture}
            \begin{axis}[xlabel=$t$, legend pos=south east,
                grid=both, xmin=0, xmax=30, no markers, ylabel=$\chi_\text{max}$,
                every axis plot/.append style={very thick}, 
                scale only axis, height=\doubleFigureHeight, width=\doubleFigureWidth, legend pos=north east]

                \addplot[color = color1]
                table[x=t, y=max_bond_dim_1em3, col sep=space]{figures/plots/HOMPS/data/homps_low_T_eps_convergence.txt};

                \addplot[color = color2]
                table[x=t, y=max_bond_dim_1em4, col sep=space]{figures/plots/HOMPS/data/homps_low_T_eps_convergence.txt};

                \addplot[color = color3]
                table[x=t, y=max_bond_dim_1em5, col sep=space]{figures/plots/HOMPS/data/homps_low_T_eps_convergence.txt};
            \end{axis}
        \end{tikzpicture}
    \end{subfigure}
    \caption{The HOMPS equations of the low temperature spin-boson model are integrated using different truncation threshholds $\epsilon_{\text{SVD}}$ for the singular value decomposition. 
    The integration method used is RK4 with a time step of $\Delta t = 0.02$, $K=13$ terms of the Padé approximation, and $N_\text{trunc} = 9$. 
    The parameters for the spin-boson model are given in the text. On the left, the stochastic
    expectation value $\mathbb{E}\left\lbrack\langle\hat{\sigma}_z\rangle\right\rbrack$ is computed over 10000 realizations. Already at $\epsilon_\text{SVD}=10^{-4}$, the method
    is well converged. On the right, the maximal bond dimension averaged over 10000 realizations is shown. Smaller truncation threshholds $\epsilon_\text{SVD}$
    lead to larger bond dimensions, giving rise to higher computation times.}
    \label{fig:homps_low_T_eps_convergence_sigma_z_and_max_bond_dim} 
\end{figure}
\newpage
\noindent Because now multiple bath nodes are used, the MPS needs to be truncated in order to keep computation times managable. In Figure \ref{fig:homps_low_T_eps_convergence_sigma_z_and_max_bond_dim}, 
the HOMPS equations are integrated using different values for the truncation threshhold $\epsilon_\text{SVD}$. After each Singular Value Decomposition,
singular values smaller than the truncation threshhold are omitted. Lower threshhold values lead to larger bond dimensions, which drastically increases computation times.
For the spin-boson model, a truncation threshhold of $\epsilon_\text{SVD} = 1.e-3$ yields good results and leads to a maximal bond dimension
of $\chi_{\text{max}} \approx 2$.
Finally, I integrate the HOMPS equations using 100, 1000, and 10000 realizations of the stochastic process in Figure \ref{fig:HOMPS_low_T_multiple_realizations}.
The parameters used are $K = 13$, $N_\text{trunc} = 9$, and $\epsilon_\text{SVD} = 10^{-3}$; The integration method is RK4. The results match well the ones obtained in \cite{Suess:2014,Song:2016}.
Note that the effect of the noise is low compared to the high temperature case (compare figure \ref{fig:homps_high_T_full_runs} and \ref{fig:HOMPS_low_T_multiple_realizations}).
\begin{figure}[!ht]
    \centering
    \begin{tikzpicture}[scale=1, trim axis left, trim axis right]
        \definecolor{color1}{HTML}{bdd7e7}
        \definecolor{color2}{HTML}{6baed6}
        \definecolor{color3}{HTML}{2171b5}
        \def\singleFigureWidth{0.5\textwidth}
        \def\singleFigureHeight{0.309\textwidth}

        \begin{axis}[xlabel=$t$, ylabel=$\mathbb{E}\left\lbrack\langle\hat{\sigma}_z\rangle\right\rbrack$,
            grid=both, xmin=0, xmax=30, ymin=-1, ymax=1, no markers, 
            every axis plot/.append style={very thick}, scale only axis, height=\singleFigureHeight, width=\singleFigureWidth]

            \addplot[color = color1]
            table[x=t, y=sigma_z_100, col sep=space]{figures/plots/HOMPS/data/homps_low_T_multiple_realizations.txt};
            \addlegendentry{100 realizations}

            \addplot[color = color2]
            table[x=t, y=sigma_z_1000, col sep=space]{figures/plots/HOMPS/data/homps_low_T_multiple_realizations.txt};
            \addlegendentry{1000 realizations}

            \addplot[color = color3]
            table[x=t, y=sigma_z_10000, col sep=space]{figures/plots/HOMPS/data/homps_low_T_multiple_realizations.txt};
            \addlegendentry{10000 realizations}

        \end{axis}
    \end{tikzpicture}   
    \caption{In this figure, the HOMPS equations of the low temperature spin-boson model are integrated using RK4 with a time step of $\Delta t = 0.02$, 
    $K=13$ terms of the Padé approximation, and $N_\text{trunc} = 9$. The parameters for the spin-boson model are given in the text.
    The stochastic expectation value is computed using 100, 1000, and 10000 realizations of the stochastic process.}
    \label{fig:HOMPS_low_T_multiple_realizations} 
\end{figure}
