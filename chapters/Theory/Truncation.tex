When integrating the HOPS equations numerically, one has to truncate the hierarchy at some order,
such that only a finite number of auxillary states $\Psi_t^{(\vectorbold{n})}$ remain. \\
The most straight-forward truncation method is to set all auxillary states for which
one entry of the index vector exceeds a certain threshhold value $N_\text{trunc}$ to zero:
\begin{equation}
    \label{eq:simple_truncation}
    \Psi_t^{(\vectorbold{n})} = 0 \quad \Leftrightarrow \quad \exists k \colon n_k \ge N_\text{trunc}.
\end{equation}
A more involved truncation method is triangular truncation, where all index vectors exceeding
a given magnitude $M_\text{trunc}$ are set to zero:
\begin{equation*}
    \Psi_t^{(\vectorbold{n})} = 0 \quad \Leftrightarrow \quad \norm{\vectorbold{n}} \ge M_\text{trunc}.
\end{equation*}
Instead of just setting truncated auxillary states to zero one can also use so-called 
\textit{terminators} for a better approximation of the exact hierarchy. Terminators for the
simple and the triangular truncation method are given in \cite{Suess:2014}. However, just setting
the truncated states to zero yields good results in practice. This is done throughout all computations
in this thesis.