The NMQSD equation (\ref{eq:non_markovian_schroedinger_equation})
cannot easily be solved numerically due to the functional derivative.
However, one can bring the equation into a hierarchically structured
set of differential equations, the Hierarchy of Pure States (HOPS) \cite{Suess:2014},
which can then be integrated numerically. In this section, we will derive the linear HOPS equation, mainly
following the derivation in \cite{Hartmann:2021}, but the same result is also obtained in \cite{Suess:2014,Hartmann:2017}.
\\
We will start by approximating the bath correlation function (BCF) (\ref{eq:bath_correlation_function}) by a finite sum of exponentials
\begin{equation*}
    \alpha\left(\tau\right) \approx \sum_{k=1}^{K} \alpha_k\left(\tau\right) \coloneqq \sum_{k=1}^{K} g_k e^{-\omega_k\tau},
\end{equation*}
with constants $g_k$ and $\omega_k$. The total number of terms $K$ corresponds
to the number of harmonic oscillators coupling to the system. Such an approximation
of the bath correlation function is possible for many systems of interest, and a specific
example is given in appendix \ref{app:Approximating_BCF_Spin_Boson}, where we approximate the BCF of the spin-boson model
using a Matsubara expansion.
\\
We will work with the discrete version of the NMQSD equation \cite{Hartmann:2021}
\begin{equation}
    \label{eq:discrete_NMQSD_equation}
    \Psi_{t+1} = \Psi_t + \Delta \cdot \left\{
        -iH_\text{S} + \hat{L}z_{t}^* - \hat{L}^\dagger \sum_{s=0}^{t-1} \alpha\left(t\Delta-s\Delta\right)\frac{\partial}{\partial z_{s}^*}
    \right\} \Psi_t,
\end{equation}
where $t, s \in \mathbb{N}_0$, we introduced the time step $\Delta$, and $z \coloneqq \left\{z_{1}, z_{2},\dots\right\}$
now is a discrete stochastic process. One can easily see that the NMQSD equation (\ref{eq:non_markovian_schroedinger_equation}) 
is recovered if the limit $\Delta\rightarrow0$ is taken. The reason for using the discrete version of the NMQSD equation
is that we can replace the functional derivative with an ordinary derivative, making the derivation much more intuitive.
\\
We define the operator
\begin{equation*}
    D_k^t \coloneqq \sum_{s=0}^{t-1} \alpha_k\left(t\Delta - s\Delta\right) \frac{\partial}{\partial z_s^*}
    = g_k \sum_{s=0}^{t-1} e^{-\omega_k\left(t-s\right)\cdot\Delta}
    \frac{\partial}{\partial z_s^*}
\end{equation*}
and the auxillary states
\begin{equation}
    \label{eq:definition_of_auxillary_states}
    \Psi_t^{(\vectorbold{n})} \coloneqq \prod_{k=1}^{K} \left(D_k^t\right)^{n_k}\Psi_t,
\end{equation}
using an index vector $\vectorbold{n}\in\mathbb{N}_0^{K}$. The physical pure state is
recovered when setting the index vector to zero, $\Psi_n = \Psi_n^{(\vectorbold{0})}$.
Using these definitions, we can rewrite the discrete NMQSD equation (\ref{eq:discrete_NMQSD_equation}):
\begin{equation*}
    \Psi_{t+1}^{(\vectorbold{0})} = \Psi_{t}^{(\vectorbold{0})} + \Delta \cdot \left(
        -iH_\text{S} + \hat{L}z_t^* - \hat{L}^\dagger\sum_{k=1}^{K}D_k^t
    \right) \Psi_{t}^{(\vectorbold{0})}
    = \Psi_{t}^{(\vectorbold{0})} + \Delta \cdot \left(
        -iH_\text{S} + \hat{L}z_t^*
    \right) \Psi_{t}^{(\vectorbold{0})} - \hat{L}^\dagger\sum_{k=1}^{K}\Psi_t^{(\vectorbold{0}+\vectorbold{e}_k)},
\end{equation*}
where $\vectorbold{e}_k$ is the $k$th unit vector.
\\
Our next goal is to derive an equation of motion for an arbitrary auxillary state $\Psi_t^{(\vectorbold{n})}$.
Using equation (\ref{eq:definition_of_auxillary_states}) we can write
\begin{equation}
    \label{eq:auxillary_state_np1}
    \Psi_{t+1}^{(\vectorbold{n})} = \prod_{k=1}^{K} \left(D_k^{t+1}\right)^{n_k} \Psi_{t+1}.
\end{equation}
We can expand
\begin{equation*}
    D_k^{t+1} = \left(1-\omega_k\cdot\Delta\right)\left(g_k\frac{\partial}{\partial z_t^*} + D_k^t\right) + O\left(\Delta^2\right)
\end{equation*}
and hence
\begin{equation*}
    \left(D_k^{t+1}\right)^{n_k} = 
    \left(1-n_k\omega_k\Delta\right)
    \left(g_k\frac{\partial}{\partial z_t^*} + D_k^t\right)^{n_k} + O\left(\Delta^2\right).
\end{equation*}
To further simplify equation (\ref{eq:auxillary_state_np1}), we can use the fact that the state
at time $t$, $\Psi^{(\vectorbold{n})}_t$, depends only on the stochastic variables $z_1, z_2, \dots, z_{t-1}$, but not
on $z_t$, which we can write as $\Psi^{(\vectorbold{n})}_t=\Psi^{(\vectorbold{n})}\left(z|_0^{t-1}\right)$. It follows that
$\frac{\partial}{\partial z_t^*}\Psi^{(\vectorbold{n})}_t = 0$. Using equation (\ref{eq:discrete_NMQSD_equation})
we can see $\frac{\partial^2}{\partial z_t^{*2}}\Psi^{(\vectorbold{n})}_{t+1} = 0$ and therefore
\begin{equation}
    \label{eq:simplification_of_D_psi_np1}
    \left(D_k^{t+1}\right)^{n_k} \Psi_{t+1} = \left(1-n_k\omega_k\Delta\right)
    \left(n_k g_k \left(D_k^t\right)^{n_k-1}\frac{\partial}{\partial z_t^*} + \left(D_k^t\right)^{n_k}\right) + O\left(\Delta^2\right).
\end{equation}
Inserting equations (\ref{eq:discrete_NMQSD_equation}) and (\ref{eq:simplification_of_D_psi_np1}) into equation (\ref{eq:auxillary_state_np1}) and performing some
additional algebra, one arrives at
\begin{equation*}
    \Psi_{t+1}^{(\vectorbold{n})} = \Psi_t^{(\vectorbold{n})} + \Delta\cdot\left(
        -iH_\text{S} - \vectorbold{n}\cdot\boldsymbol{\omega} + \hat{L}z_t^*
    \right) \Psi_t^{(\vectorbold{n})}
    + \Delta \cdot \hat{L}\sum_{k=1}^{K}n_kg_k\Psi_t^{(\vectorbold{n}-\vectorbold{e}_k)}
    - \Delta \cdot \hat{L}^\dagger\sum_{k=1}^{K}\Psi_t^{(\vectorbold{n}+\vectorbold{e}_k)}.
\end{equation*}
Performing the limit $\Delta \rightarrow 0$, we obtain the linear HOPS equation for a system coupled to $K$ bath modes:
\begin{equation}
    \label{eq:linear_HOPS_single_site}
    \frac{\partial}{\partial t}\Psi_t^{(\vectorbold{n})} = \left(
        -iH_\text{S} - \vectorbold{n}\cdot\boldsymbol{\omega} + \hat{L}z_t^*
    \right) \Psi_t^{(\vectorbold{n})}
    + \hat{L}\sum_{k=1}^{K}n_kg_k\Psi_t^{(\vectorbold{n}-\vectorbold{e}_k)}
    - \hat{L}^\dagger\sum_{k=1}^{K}\Psi_t^{(\vectorbold{n}+\vectorbold{e}_k)}.
\end{equation}
Note that the HOPS equations do not contain any functional derivatives and therefore can be readily integrated numerically.