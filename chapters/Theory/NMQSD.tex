To simulate an open quantum system, we first need to model the system, its environment, and the interaction between the two.
We will consider a system \textit{S} linearly coupled to a bath \textit{B} of harmonic oscillators. 
We can split the Hamiltonian of such a model into a system, bath, and interaction part
\begin{equation*}
    \hat{H} = \hat{H}_\text{S} \otimes \mathbbm{1}_\text{B} + \mathbbm{1}_\text{S} \otimes \hat{H}_\text{B}
    + \hat{H}_\text{int}.
\end{equation*}
We assume that the bath consists of $K$ harmonic oscillators, which couple linearly to the system. 
The bath Hamiltonian is then given by
\begin{equation*}
    \hat{H}_\text{B} = \sum_{k=1}^{K}\nu_{k} 
    \hat{a}^\dagger_{k} \hat{a}_{k},
\end{equation*}
where $\hat{a}^\dagger_{k}$, $\hat{a}_{k}$ are the bosonic creation and annihilation
operators of the $k$th harmonic oscillator, and $\nu_{k}$ are constants.
The interaction Hamiltonian can be written as
\begin{equation*}
    \hat{H}_\text{int} = \sum_{k=1}^{K} \left( \gamma_{k}^*
    \hat{L} \otimes \hat{a}_{k}^\dagger + \text{h.c.} \right)
\end{equation*}
with constants $\gamma_{k}$. The system operator $\hat{L}$ describes the coupling
of the system to the bath modes.
\\
In the context of open systems it is useful to 
define the \textit{bath correlation function}
\begin{equation}
    \label{eq:bath_correlation_function}
    \alpha(\tau) = \frac{1}{\pi} \int_0^\infty \text{d}\omega S(\omega) 
    \left[\coth\left(\frac{\omega}{2T}\right)\cos\left(\omega\tau\right)-i\sin\left(\tau\right)\right]
\end{equation}
with the \textit{spectral density} $S\left(\omega\right)$. The bath correlation function fully
characterizes the influence of the environment at temperature T \cite{Gao:2022} and is connected
to the constants $\nu_{k}$ and $\gamma_{k}$.
\\
We are interested in the dynamics of the system S, which can be described in terms of the reduced
density matrix
\begin{equation*}
    \rho\left(t\right) = \trace_\text{B}\left\{\rho_\text{tot}\left(t\right)\right\},
\end{equation*}
where $\rho_\text{tot}\left(t\right)$ is the density matrix of the total system (system and bath) at time $t$.
$\trace_\text{B}\left\{\cdots\right\}$ denotes the trace over all bath degrees of freedom.
We assume that the total system is initially in the state
\begin{equation*}
    \rho_\text{tot}\left(0\right) = \rho_\text{S}\left(0\right) \otimes \rho_\text{B, therm},
\end{equation*}
where the bath is in the thermal state
\begin{equation*}
    \rho_\text{therm}^{B} = \frac{e^{-\hat{H}_\text{B} / T}}{Z_\text{B}}
\end{equation*}
with the partition function $Z_\text{B} = \trace_\text{B}\left\{e^{-\hat{H}_\text{B} / T}\right\}$.
\\
The idea of Non-Markovian Quantum State Diffusion (NMQSD) \cite{Diosi:1998,Diosi:1997,Strunz:1999,Percival:1999,Strunz:1996} is that one can obtain the reduced density matrix
$\rho\left(t\right)$ from an average over pure states
\begin{equation}
    \label{eq:averaging_states_linear}
    \rho\left(t\right) = \mathbb{E}\left[\ket*{\Psi_t\left(z\right)}\bra*{\Psi_t\left(z)\right)}\right].
\end{equation}
The pure states $\ket*{\Psi_t\left(z\right)} \in \mathcal{H}_\text{S}$ are vectors in the system
Hilbert space $\mathcal{H}_\text{S}$ that depend on a complex gaussian stochastic process $z \colon t \rightarrow z_t \in \mathbb{C}$.
The expectation value $\mathbb{E}\left[\cdots\right]$ can then be computed by taking the
average over different realizations of $z$. For NMQSD, the stochastic process must
have the following properties:
\begin{equation}
    \label{eq:stochastic_process_condition}
    \begin{aligned}
        & \mathbb{E}\left[z_{t}\right]=\mathbb{E}\left[z_{t}^*\right]=0,\\
        & \mathbb{E}\left[z_{t}z_{s}\right]=0,\\
        & \mathbb{E}\left[z_{t}z_{s}^*\right]=\alpha\left(t-s\right). 
    \end{aligned}
\end{equation}
Each pure state $\ket*{\Psi_t\left(z\right)}$ starts off in the same initial state
$\ket*{\Psi_{t=0}\left(z\right)}=\ket*{\Psi_0}$ and then evolves according
to the Non-Markovian Quantum State Diffusion (NMQSD) equation \cite{Diosi:1997,Diosi:1998}
\begin{equation}
    \label{eq:non_markovian_schroedinger_equation}
    \frac{\partial}{\partial t} \ket*{\Psi_t} = -i\hat{H}_\text{S} \ket*{\Psi_t}
    + \hat{L} z_{t}^* \ket*{\Psi_t}
    - \hat{L}^\dagger \int_0^t \text{ds} \, \alpha\left(t-s\right) 
    \frac{\delta \ket*{\Psi_t}}{\delta z_{s}^*},
\end{equation}
where we omitted the explicit dependency of $\ket*{\Psi_t}$ on $z$ due to brevity.
It is important to realize that the NMQSD equation describes the dynamics in terms of a stochastic
expectation value of pure states, whereas regular master equations involve a non-stochastic
differential equation of the reduced density matrix. The advantage of the NMQSD equation is that
it is often easier to work with pure states than with density matrices.