There are multiple options to generate gaussian stochastic processes with the properties (\ref{eq:stochastic_process_condition}).
In the following, I will give three examples, which can all also be found in my repository \cite{Sappler:2023}. \\
Firstly, one can generate the process using a complex multivariate normal distribution, which is e.g. implemented in
the python library \verb|numpy|. The disadvantage of this method is that computing the multivariate gaussian is slow for
large stochastic processes, which are necessary if we want to compute HOPS with a small time step. \\
Second, one can use the method
discussed in the appendix of \cite{Song:2016} and in the supplementary material of \cite{Gao:2022}, where the integral in the
bath correlation function is approximated with a sum. This method generally works well, but introduces two additional parameters to tune 
(the cutoff and step size for approximating the integral with a sum). 
Lastly, one can use a \textit{fourier filtering} technique \cite{Ojalvo:1994}. The idea of this method is to first generate uncorrelated gaussian
white noise, transform it to the frequency domain using a fourier transform, and multiplying with a filter. An inverse fourier transform is
then used to go back to the time domain, creating noise with the requested correlations. \\
Throughout my thesis, the fourier filtering technique is used to generate stochastic processes. In the following, I will explain
the method in more detail. \\
We start by generating complex gaussian white noise. This can for example be done with the \textbf{Box-Mueller-Wiener algorithm}:
Given two random numbers $\xi_1, \xi_2 \in[0, 1]$ drawn from a uniform distribution, we can obtain the complex white noise $\theta$ as
\begin{equation*}
    \theta = \sqrt{-\log(\xi_1)} \cdot e^{2i\pi\xi_2}.
\end{equation*}
Our objective now is to generate a discrete stochastic process z of length $N$ with correlations
\begin{equation*}
    \left\langle z_t z^*_s\right\rangle = \alpha(t - s) \equiv \alpha(\tau).
\end{equation*}
For this, we transform the white noise $\theta_t = \left\{\theta_1, \theta_2, \dots, \theta_{N}\right\}$ into frequency space:
\begin{equation*}
    \hat{\theta}_k = \sum_{t=1}^{N} \theta_t e^{-\frac{2\pi i}{N}kt}.
\end{equation*}
Next, we construct the correlated noise in frequency space
\begin{equation*}
    \hat{z}_k \coloneqq \hat{\theta}_k \cdot \sqrt{\hat{\alpha}_k},
\end{equation*}
where we have introduced the fourier transformed bath correlation function
\begin{equation*}
    \hat{\alpha}_k = \sum_{n = 0}^{N} \alpha(n\cdot\Delta t) e^{-\frac{2\pi i}{N} k\Delta t}
\end{equation*}
with the time step $\Delta t$. To obtain the correlated noise in the time domain, we simply perform an inverse fourier transform
\begin{equation*}
    z_t = \frac{1}{N} \sum_{k = 1}^{N} \hat{z}_k.
\end{equation*}
One can show that the stochastic process $z$ fulfills the conditions (\ref{eq:stochastic_process_condition}).
It is important to note that, because of the symmetries of the Fourier transform, only $N/2$ of the generated
values can be used; the other half are periodically correlated with the first half. 