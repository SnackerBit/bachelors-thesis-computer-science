A problem of the linear HOPS is that the states are not normalized. For that reason, different realizations of the noise can
produce state vectors with vastly different magnitudes. The stochastic expectation value is then dominated by the state vectors
with the largest magnitudes, which causes the expectation value to converge very slowly.
If a specific realization happens to produce a state vector with low magnitude, this will not change the result much and can be seen
as wasted computation time. To fix this problem, one can derive a non-linear version of HOPS, where the density matrix of the
reduced systems is computed as an expectation value over normalized states
\begin{equation*}
    \widetilde{\Psi}_t(z) \coloneqq \frac{\Psi_t(z)}{\norm{\Psi_t(z)}}
\end{equation*}
instead:
\begin{equation}
    \label{eq:averaging_states_nonlinear}
    \rho\left(t\right) = \mathbb{E}\left[\ket*{\widetilde{\Psi}_t\left(z\right)}\bra*{\widetilde{\Psi}_t\left(z)\right)}\right].
\end{equation}
 The non-linear HOPS equations are obtained by replacing \cite{Diosi:1998}
\begin{equation*}
    \hat{L}^\dagger \rightarrow \hat{L}^\dagger - \left\langle
        \hat{L}^\dagger
    \right\rangle_t
\end{equation*}
and
\begin{equation}
    \label{eq:memory_term_nonlinear_HOPS}
    z_t^* \rightarrow \tilde{z}_t^* \coloneqq z_t^* + \int_0^t \alpha^*(t-s) \left\langle
        \hat{L}^\dagger
    \right\rangle_s \text{ds}
\end{equation}
in the linear HOPS equations. Here, $\left\langle \cdot \right\rangle_t$ denotes the expectation value at time $t$.
The full non-linear HOPS then becomes
\begin{equation}
    \label{eq:non_linear_HOPS_single_site}
    \frac{\partial}{\partial t}\Psi_t^{(\vectorbold{n})} = \left(
        -i\hat{H}_\text{S} - \vectorbold{n}\cdot\boldsymbol{\omega} + \hat{L}\tilde{z}_t^*
    \right) \Psi_t^{(\vectorbold{n})}
    + \hat{L}\sum_{k=1}^{K}n_kg_k\Psi_t^{(\vectorbold{n}-\vectorbold{e}_k)}
    - \left(
        \hat{L}^\dagger - \langle
        \hat{L}^\dagger
    \rangle_t
    \right)\sum_{k=1}^{K}\Psi_t^{(\vectorbold{n}+\vectorbold{e}_k)}.
\end{equation}