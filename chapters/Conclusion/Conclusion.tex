As we have seen, the HOPS method is able to relieably simulate the dynamics of non-Markovian open quantum systems. 
If one wants to use multiple bath modes, the HOMPS method is a good strategy to minimize memory requirements by
compressing the quantum state into Matrix Product form. We discussed two integration methods, RK4 and TDVP2, which are
both viable for solving the HOMPS equations. There are however some additional steps that can be made to further
improve the methods. \\
First, there exists an alternative realization of the non-Markovian stochastic Schrödinger equation \cite{Song:2016},
where, in contrast to (\ref{eq:stochastic_process_condition}), a stochastic process with non-zero correlations 
$\mathbb{E}\left[z_{t}z_{s}\right]\neq0$ is used. This leads to a smaller bath correlation function, reducing the
amplitude of the non-Markovian memory term at high temperatures. For models that need many terms for approximating the 
bath correlation function sufficiently well, this can be a crucial improvement.\\
Second, one has to consider how systems consisting of multiple subsystems, i.e., many-body systems, should be treated in the HOMPS method. Because of the
exponential growth of the Hilbert space, using one tensor for the complete many-body system is inefficient. It is therefore
better to split the system into smaller subsystems and to represent each subsystem by a single tensor. A question that arises
is how to connect the different tensors. One idea is to use an MPS, repeating a structure where each physical tensor is followed
by multiple bath mode tensors, which is done in \cite{Gao:2022}. Alternatively, it could be beneficial to use a tree tensor network \cite{Holzner:2010}
instead. The physical subsystems would then be represented by rank-4 tensors, where two legs are used to connect to the neighbouring subsystems,
one leg is the physical leg, and the last leg is connected to an MPS representing the bath modes.
TDVP2 can be adapted for tree tensor networks \cite{Holzner:2010}. This approach would help to further reduce the
memory requirements for computing the non-Markovian dynamics of open many-body systems.\\
In conclusion, I believe that the HOPS and HOMPS methods are very useful for simulating open quantum systems and will be widely used
for studying new systems and comparing experiments to theory in the future.