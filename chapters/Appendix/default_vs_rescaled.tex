The non-rescaled HOMPS (\ref{eq:HOMPS_MPO}) has numerical stability problems for higher values of $N_\text{trunc}$. To illustrate the problem,
I plot the expectation value $\mathbb{E}\left\lbrack\langle\hat{\sigma}_z\rangle_t\right\rbrack$ of the high temperature spin-boson model in Figure \ref{fig:default_vs_rescaled_sigma_z}, using both
the non-rescaled HOMPS (\ref{eq:HOMPS_MPO}) and the rescaled HOMPS (\ref{eq:HOMPS_MPO_rescaled}). For a better comparison, the stochastic process
was set to $z(t) = 0$ for both methods. A high truncation value of $N_\text{trunc}=40$ was used. One can see that the non-rescaled HOMPS diverges
after a short time and does not produce the correct behaviour. \\
\begin{figure}[!ht]
    \centering
    \begin{tikzpicture}[scale=1, trim axis left, trim axis right]
        \definecolor{color1}{HTML}{4477AA}
        \definecolor{color2}{HTML}{EE6677}
        \def\singleFigureWidth{0.5\textwidth}
        \def\singleFigureHeight{0.309\textwidth}

        \begin{axis}[xlabel=$t$, ylabel=$\mathbb{E}\left\lbrack\langle\hat{\sigma}_z\rangle\right\rbrack$,
            grid=both, xmin=0, xmax=30, ymin=0, ymax=1, no markers, 
            every axis plot/.append style={very thick}, scale only axis, height=\singleFigureHeight, width=\singleFigureWidth,
            legend pos=south west]

            \addplot[color = color1]
            table[x=t, y=sigma_z_default, col sep=space]{figures/plots/Appendix_B/data/rescaled_homps_sigma_z.txt};
            \addlegendentry{non-rescaled HOMPS}

            \addplot[color = color2, dashed]
            table[x=t, y=sigma_z_rescaled, col sep=space]{figures/plots/Appendix_B/data/rescaled_homps_sigma_z.txt};
            \addlegendentry{rescaled HOMPS}

        \end{axis}
    \end{tikzpicture}   
    \caption{The expectation value $\mathbb{E}\left\lbrack\langle\hat{\sigma}_z\rangle\right\rbrack$ is computed for both the non-rescaled and the
    rescaled HOMPS. I use the high temperature spin-boson model with the same parameters as in Figure \ref{fig:homps_high_T_full_runs}, but only compute a single realization whithout noise ($z(t) = 0$).
    One can see that the non-rescaled version of HOMPS is unstable, diverging after a short time.}
    \label{fig:default_vs_rescaled_sigma_z} 
\end{figure}
\newline
\noindent The reason for this instability could be the fact that different auxillary states have vastly different magnitudes in the non-rescaled
version of HOMPS. In Figure \ref{fig:default_vs_rescaled_magnitudes}, I plot the magnitudes of the different auxillary states of both the non-rescaled and rescaled HOMPS against 
time. One can see that the non-rescaled version produces magnitudes that differ by up to 15 orders of magnitudes, whereas in the rescaled
version they differ only up to 7 orders of magnitude. In the HOMPS method addition of tensors is performed, which could lead to cancellation,
a well-known limitation of floating point arithmetic. The rescaling "normalizes" the auxillary states and increases numerical stability. 
\begin{figure}[!ht]
    \def\doubleFigureWidth{5cm}
    \def\doubleFigureHeight{4.635cm}
    \centering
    \begin{subfigure}[b]{0.34\textwidth}
        \begin{tikzpicture}[scale=1]
            \begin{axis}[xlabel=$t$, ylabel=$||\Psi^{(n)}_t||$,
                grid=both, xmin=0, xmax=30, ymin=1e-16, ymax=1, no markers, 
                every axis plot/.append style={very thick}, 
                scale only axis, height=\doubleFigureHeight, width=\doubleFigureWidth, ymode=log, title=Not Rescaled,
                point meta min=1, point meta max=40,
                cycle list={[samples of colormap={40 of colormap/Blues-3}]}]

                \foreach \x in {0,...,39}{
                    \addplot table [col sep=space, x=t, y=\x] {figures/plots/Appendix_B/data/auxillary_magnitudes_default.txt};
                }


            \end{axis}
        \end{tikzpicture}
    \end{subfigure}
    \centering
    \begin{subfigure}[b]{0.64\textwidth}
        \centering
        \begin{tikzpicture}
            \begin{axis}[xlabel=$t$, yticklabels={,,}, legend pos=south east,
                grid=both, xmin=0, xmax=30, ymin=1e-16, ymax=1, no markers, 
                every axis plot/.append style={very thick}, 
                scale only axis, height=\doubleFigureHeight, width=\doubleFigureWidth, legend pos=north east, ymode=log, title = Rescaled,
                cycle list={[samples of colormap={40 of colormap/Blues-3}]},
                colorbar, colormap/Blues-3, 
                colorbar style={
                    title=$n$,
                    yticklabels={0,10,20,30,40},
                    ytick = {0,0.25,0.5,0.75,1},
                },
                point meta max = 1,
                point meta min = 0]

                \foreach \x in {0,...,39}{
                    \addplot table [col sep=space, x=t, y=\x] {figures/plots/Appendix_B/data/auxillary_magnitudes_rescaled.txt};
                }

            \end{axis}
        \end{tikzpicture}
    \end{subfigure}
    \caption{In this figure, the magnitudes of different HOMPS auxillary states $\Psi^{(n)}_t$ are plotted against $t$. I use the high temperature
    spin-boson model with the same parameters as in Figure \ref{fig:homps_high_T_full_runs}, but only compute a single realization whithout noise ($z(t) = 0$).
    One can see that the magnitudes of different auxillary states take on vastly more widespread values when using the non-rescaled version of
    HOMPS than when using the rescaled version.}
    \label{fig:default_vs_rescaled_magnitudes} 
\end{figure}